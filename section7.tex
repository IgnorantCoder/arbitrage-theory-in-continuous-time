\documentclass{jsbook}

\usepackage{amsmath}
\usepackage{amssymb}
\usepackage{amsthm}

\newtheoremstyle{withoutdotendstyle}% % Name
  {}%                      % Space above
  {}%                      % Space below
  {\normalfont}%           % Body font
  {}%                      % Indent amount
  {\bfseries}%             % Theorem head font
  {}%                      % Punctuation after theorem head
  { }%                     % Space after theorem head, ' ', or \newline
  {}%                      % Theorem head spec (can be left empty, meaning `normal')
\theoremstyle{withoutdotendstyle}

\newtheorem{theorem}{定理}[chapter]
\newtheorem{definition}[theorem]{定義}
\newtheorem{lemma}[theorem]{補題}
\newtheorem{assumption}{仮定}[section]

\numberwithin{theorem}{chapter}% Reset theorem counter with every chapter

\renewcommand\proofname{\bf 証明}

\begin{document}
\setcounter{chapter}{6}

\chapter{裁定価格}
\section{導入}
\begin{definition}
  価格過程$B$が無リスク資産の価格であるとは、
  \begin{align}
    dB\left(t\right)=r\left(t\right)B\left(t\right)dt
  \end{align}
  と書けることである。$r\left(t\right)$は任意の適合過程とする。
\end{definition}
\section{条件付き請求権と裁定}
\begin{definition}

\end{definition}
\begin{definition}

\end{definition}
\begin{definition}

\end{definition}
\begin{definition}
金融市場における{\bf 裁定} 可能性とは 自己調達的なポートフォリオ$h$が以下を満たすことである。
\begin{align}
  V^{h}\left(0\right)&=0\\
  P\left(V^{h}\left(T\right)\geq 0\right)&=1\\
  P\left(V^{h}\left(T\right)> 0\right)&>1
\end{align}
もし裁定の可能性がない場合、市場は無裁定であるという。
\end{definition}
\end{document}
